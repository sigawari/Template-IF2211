% !TeX root = thesis.tex
%--------------------------------------------------------------------%
%
% Template Laporan Tugas Besar IF2211
% Format Terinspirasi dari TA LaTeX Teknik Informatika ITERA.
% Editor: Radhinka Bagaskara, Martin C.T. Manullang, iwawiwi
% Version 2025.3
% https://github.com/rdhnk/Latex-TA-IF-ITERA/blob/main/thesis.tex
% dan direkayasa ulang oleh Sikah N. Ilmi
% Semua tabel, gambar, dll. itu hanya CONTOH silahkan disesuaikan dengan analisis, perhitungan, dan deskripsi Anda sendiri
%--------------------------------------------------------------------%

% Set jenis dokumen Tugas Akhir
\documentclass[12pt, a4paper, onecolumn, oneside, final]{report}

\input{config/config.sty}

\bibliography{references}

\renewcommand\subsubsection{%
  \@startsection{subsubsection}{3}
  {-2pc}% <------- The opposite of what we set for subs
  {-3.25ex\@plus -1ex \@minus -.2ex}%
  {1.5ex \@plus .2ex}%
  {\normalfont\normalsize\bfseries}}
\makeatother

\begin{document}
% 	\pagestyle{plain}
% 	\fancyhf{}
% 	\rfoot{Halaman \thepage}%

    %----------------------------------------------------------------%
    % Konfigurasi Informasi Tugas Akhir
    %----------------------------------------------------------------%
    
    % Judul Tugas Akhir
    \title{IMPLEMENTASI ALGORITMA X DALAM PEMECAHAN BOT PERMAINAN DIAMOND (contoh judul)} 	
    % DITULIS DALAM HURUF KAPITAL; Font size 14 pt; Bold; Tidak melebihi 3 baris
    \titleEN{Judul Tugas Besar (Inggris)}      % Judul Tugas Besar dalam Bahasa Inggris
    

	\sloppy % mencegah text overflow. (Jose)
    \pagenumbering{roman}
    \setcounter{page}{1} % Nomor halaman dimulai dengan "ii" di hal. Pengesahan

    \clearpage
\pagestyle{empty}

\begin{center}
    \smallskip

    {\fontsize{14pt}{16pt}\selectfont\bfseries \thetitle}

    \bigskip

    {\fontsize{14pt}{16pt}\selectfont\bfseries Tugas Besar}

    \medskip

    {\fontsize{12pt}{14pt}\selectfont
    Diajukan sebagai syarat menyelesaikan mata kuliah Strategi Algoritma (IF2211) Kelas RX di Program Studi Teknik Informatika, Fakultas Teknologi Industri, \mbox{Institut Teknologi Sumatera}}

    \vfill

    \begin{figure}[h]
        \centering
        \includegraphics[width=7cm, keepaspectratio]{figure/fototiga.jpg}
    \end{figure}

    \medskip

    {\fontsize{12pt}{14pt}\selectfont\bfseries Oleh: Kelompok X (Nama Kelompok)}

    {\fontsize{12pt}{14pt}\selectfont
    \begin{tabular}{ll}
        Meazza Mei Aza & 122140345 \\
        Nasnas Lalalala & 122140543 \\
        Tiki Taka Toe & 122140666 \\
    \end{tabular}
    }
    
    \bigskip
    
    {\fontsize{12pt}{14pt} Dosen Pengampu: Winda Yulita, M.Cs./Imam Ekowicaksono, S.Si., M.Si. (sesuaikan)
    }

    \vfill
    
    {\fontsize{14pt}{16pt}\selectfont\bfseries\MakeUppercase{
    Program Studi Teknik Informatika \\
        Fakultas Teknologi Industri\\
        Institut Teknologi Sumatera\\
        Lampung Selatan\\
    } 
        \the\year
    }

\end{center}

\clearpage % Hardcover

	% Daftar Isi
	\phantomsection% 
	\addcontentsline{toc}{chapter}{Daftar Isi}
    \tableofcontents
    \pagebreak
    
%    \input{chapters/symbols}

    %----------------------------------------------------------------%
    % Konfigurasi Bab
    %----------------------------------------------------------------%
    \renewcommand{\chaptername}{BAB}
    % Bab: Arabic
    \renewcommand{\thechapter}{\Roman{chapter}}
    % Sub-bab: Roman
    \renewcommand\thesection{\arabic{chapter}.\arabic{section}}
    
    % Setting supaya nomor halaman pertama dengan "chapter"
    % berada di tengah bawah, tapi selanjut2nya di kanan atas
    \fancypagestyle{plain}{%
    	\fancyhf{}%
    	\renewcommand{\headrulewidth}{0pt}
    	\fancyhead[]{}
    	\fancyfoot[C]{\thepage}
    }
    % Reset penomoran halaman menjadi 1
    \setcounter{page}{1}
    \pagenumbering{arabic}
    
    % Set spasi antar paragraf menjadi 0pt
    \setlength{\parskip}{0pt}
    
    %----------------------------------------------------------------%

    %----------------------------------------------------------------%
    % Daftar Bab
    % Untuk menambahkan daftar bab, buat berkas bab misalnya `chapter-6` di direktori `chapters`, dan masukkan ke sini.
    %----------------------------------------------------------------%
    \justifying
    \newpage
\chapter{Deskripsi Tugas} \label{Bab I}

Mulai deskripsi tugasnya disini. Deskripsi tugas berisi rincian tugas yang dikemukakan dalam bentuk paragraf. Boleh melakukan copy paste dari deskripsi tugas di spesifikasi tugas besar boleh diringkas (lebih baik dijelaskan dengan bahasamu sendiri). Boleh juga dijelaskan menggunakan sub-bab agar lebih terstruktur seperti:
\section{Latar Belakang}
Menjelaskan pentingnya penyusunan strategi algoritma dalam pembuatan bot otomatis, khususnya dalam konteks permainan Diamonds, serta penerapan algoritma Greedy untuk mencapai tujuan maksimal. \par
\section{Tujuan Tugas Besar}
\section{Ruang Lingkup Tugas}
\section{Spesifikasi Tugas}

    \newpage

\chapter{Landasan Teori} \label{Bab II}

\section{Dasar Teori} \label{II.Dasar Teori}
Jelaskan algoritma greedy secara umum. Boleh menggunakan referensi dari luar, jangan lupa disitasi dengan format IEEE. \par

Perujukan literatur dapat dilakukan dengan menambahkan entri baru dalam file \verb|references.bib|. Cara merujuk sitasi menggunakan \verb|\cite{nama label sitasi}|. Hasil sitasi seperti ini: \cite{knuth2001art}. Daftar Pustaka hanya akan memunculkan sitasi yang direferensikan menggunakan command \verb|\cite{}|. \par

\section{Cara Kerja Program} \label{II.Cara_Kerja}
Bagaimana cara kerja program secara umum (bagaimana bot melakukan aksinya, bagaimana mengimplementasikan algoritma greedy ke dalam bot, bagaimana menjalankan bot, dll). Bagian ini boleh diisi bagian deskripsi umum cara kerjanya saja kemudian diperinci per sub-bagian lagi.

\subsection{Cara Implementasi Program}
Cara kerja program ini adalah dengan mengimplementasikan algoritma greedy untuk menyelesaikan masalah yang ada. Algoritma ini akan melakukan pemilihan item berdasarkan kriteria tertentu yang sudah ditentukan sebelumnya. Setiap item akan diperiksa dan dimasukkan ke dalam solusi jika memenuhi kriteria, sampai seluruh item diproses. Program ini juga menggunakan struktur data yang efisien untuk menyimpan dan memproses input, seperti array atau list.

\bigskip

\subsection{Menjalankan Bot Program}
Untuk menjalankan bot program, langkah pertama adalah memastikan bahwa semua kebutuhan library dan dependensi telah terinstal. Beberapa library yang diperlukan untuk menjalankan bot ini termasuk [sebutkan nama library], serta pengaturan API yang diperlukan untuk berinteraksi dengan platform yang relevan. Berikut adalah tahapan-tahapan untuk menjalankan bot:
    \begin{enumerate}
        \item Pastikan API key sudah dikonfigurasi dengan benar.
        \item Lakukan instalasi library yang dibutuhkan menggunakan pip, misalnya: \texttt{pip install library-name}.
        \item Jalankan bot dengan command tertentu atau melalui script Python yang sudah dipersiapkan.
        \item Pastikan bot berfungsi dengan memverifikasi status atau melakukan testing sederhana.
    \end{enumerate}

\bigskip

\subsection{Fitur Tambahan atau Aspek Lain}
 Jika ada fitur tambahan atau aspek lain terkait cara kerja bot, misalnya pengaturan tambahan untuk memonitor performa atau menambah fitur baru seperti [sebutkan fitur], ini bisa dijelaskan di sini. Selain itu, dapat juga dijelaskan mengenai pengaturan environment atau penyesuaian lainnya agar bot dapat beroperasi dengan optimal.

    \newpage
\chapter{Aplikasi Strategi \textit{Greedy}} \label{Bab III}

\section{Proses \textit{Mapping}} \label{III.Mapping}
Digambarkan terkait bagaimana proses mapping yang dilakukan dalam game. Semisalkan butuh untuk menjelaskan per langkah, bisa menggunakan penomoran biasa:\par
\begin{enumerate}[noitemsep]
    \item \textit{Indexing} dengan spesifikasi minimum sistem operasi Windows 11, processor AMD Ryzen 5 7430 CPU @ 6 core/2,3 GHz, RAM 16GB DDR4
    \item Github
    \item Ini cuma contoh ya
\end{enumerate}

\section{Eksplorasi Alternatif Solusi Greedy} \label{III.EksplorAlternatif}
Jelaskan alternatif solusi greedy yang kalian pikirkan misalkan algoritma abcd. \par

\section{Analisis Efisiensi dan Efektivitas Solusi Greedy} \label{III.AnalisisEfisiensi}
Kemudian lakukan analisis terkait alternatif-alternatif yang kamu lakukan di sub-bab sebelumnya di sini. Boleh menggunakan tabel seperti:
\par

\begin{table}[H]
\renewcommand{\arraystretch}{1.5} % Menambah jarak vertikal antar baris
\centering
\caption{Perbandingan Alternatif Solusi Greedy}
\begin{tabular}{|c|c|c|c|}
\hline
\textbf{Algoritma} & \textbf{Kecepatan Eksekusi} & \textbf{Akurasi Solusi} & \textbf{Kompleksitas} \\ \hline
Algoritma X        & Cepat                        & Tinggi                   & Rendah                \\ \hline
Algoritma Y        & Sedang                       & Sedang                   & Sedang                \\ \hline
Algoritma Z        & Lambat                       & Tinggi                   & Tinggi                \\ \hline
\end{tabular}
\label{tab:perbandingan_greedy}
\end{table}

\subsection{Strategi Greedy yang Dipilih} \label{III.Strategi Dipilih}
Dari sekian banyak alternatif yang kalian pikirkan sebelumnya dan telah kalian analisis, disini tulislah strategi yang akhirnya kamu pilih. \par
    \newpage
\chapter{Implementasi dan Pengujian} \label{Bab IV}

\section{Implementasi Algoritma \textit{Greedy}} \label{IV.Algoritma}

\subsection{Pseudocode}

Berikut adalah pseudocode dari algoritma \textit{Greedy} yang digunakan dalam program:

\begin{quote}
\begin{verbatim}
function GreedyAlgorithm(input):
    sort input based on specific criteria
    solution = empty list
    for each item in input:
        if item is feasible:
            add item to solution
    return solution
\end{verbatim}
\end{quote}

\subsection{Penjelasan Alur Program}\label{IV.Alur}

Program ini mengimplementasikan algoritma greedy untuk mencari solusi optimal. Berikut adalah langkah-langkah dari alur program:

\begin{enumerate}
    \item Program mengurutkan data input berdasarkan kriteria tertentu, seperti nilai terbesar atau terkecil, tergantung pada permasalahan yang diselesaikan.
    \item Setelah data diurutkan, program memeriksa setiap item satu per satu untuk menentukan apakah item tersebut memenuhi syarat (feasible) untuk ditambahkan ke dalam solusi.
    \item Jika item memenuhi syarat, maka item tersebut ditambahkan ke dalam daftar solusi.
    \item Proses pemeriksaan berlanjut hingga semua item dalam input selesai diproses.
    \item Setelah selesai, solusi akhir dikembalikan sebagai hasil akhir dari algoritma.
\end{enumerate}

Langkah-langkah ini memastikan bahwa solusi yang diperoleh sesuai dengan prinsip greedy, yaitu mengambil keputusan terbaik pada setiap langkah.

\section{Struktur Data yang Digunakan} \label{IV.Struktur Data}

Untuk mendukung implementasi algoritma greedy, digunakan beberapa struktur data berikut:

\begin{enumerate}
    \item \textbf{Array/List}: Digunakan untuk menyimpan dan mengelola data input yang akan diproses.
    \item \textbf{Queue/Stack (Opsional)}: Digunakan bila diperlukan dalam pemrosesan lanjutan untuk mengatur urutan elemen.
    \item \textbf{Boolean Flags}: Digunakan untuk memverifikasi apakah suatu item memenuhi kriteria dan dapat dimasukkan ke dalam solusi.
\end{enumerate}

Struktur data ini memungkinkan program berjalan secara efisien dalam memproses data berdasarkan strategi greedy.

\section{Pengujian Program} \label{IV.Pengujian Program}

\subsection{Skenario Pengujian}

Pengujian dilakukan untuk membandingkan performa dua metode algoritma greedy yang berbeda. Berikut adalah hasil pengujian:

\vspace{1em} % space before the table
\begin{longtable}{|c|c|c|}
	\caption{Contoh Hasil Pengujian}
	\label{table:4.}\\
	\hline
	\textbf{Pengujian} & \textbf{Metode A} & \textbf{Metode B} \\
	\hline
	\endhead
	Kecepatan & 10 ms & 12 ms \\ 
	\hline
	Memori & 10 MB & 7 MB \\
	\hline
\end{longtable}
\vspace{1em} % space after the table

Pada Tabel \ref{table:4.}, terdapat dua indikator yang dibandingkan, yaitu kecepatan dan penggunaan memori. Metode A memiliki kecepatan lebih baik (10 ms) dibandingkan Metode B (12 ms). Namun, Metode B lebih hemat memori dengan hanya menggunakan 7 MB, sementara Metode A menggunakan 10 MB.

\subsection{Hasil Pengujian dan Analisis}

Berdasarkan hasil pengujian, berikut analisis dari masing-masing metode:

\begin{enumerate}
    \item \textbf{Kecepatan}: Metode A unggul dari sisi waktu dengan perbedaan 2 ms. Ini mengindikasikan efisiensi lebih tinggi dalam hal eksekusi.
    \item \textbf{Penggunaan Memori}: Metode B menunjukkan efisiensi dalam penggunaan memori, menjadikannya lebih ideal untuk sistem dengan keterbatasan memori.
    \item \textbf{Kesimpulan}: Pemilihan metode tergantung pada kebutuhan aplikasi. Metode A cocok jika waktu respons lebih penting, sedangkan Metode B lebih sesuai jika efisiensi memori menjadi prioritas.
\end{enumerate}

    \newpage
\chapter{Kesimpulan dan Saran} \label{Bab V}

\section{Kesimpulan} \label{V.Kesimpulan}
Berisi kesimpulan dari hasil dan pembahasan terkait penelitian yang dilakukan, dapat juga berupa temuan yang Anda dapatkan setelah melakukan penelitian atau analisis terhadap tugas besar Anda.

\section{Saran} \label{V.Saran}
Berisi saran mengenai aspek Tugas Besar atau temuan dalam penelitian/pembuatan program. Diutamakan saran berdasarkan hasil analisis dari Subbab \ref{IV.Pengujian Program}. Saran dapat dikembangkan dan diperkaya untuk Tugas Besar selanjutnya. 
    %----------------------------------------------------------------%

    % Daftar Pustaka
    \newpage
    \phantomsection% 
    \addcontentsline{toc}{chapter}{Daftar Pustaka}
    \printbibliography[title={Daftar Pustaka}]

    % Lampiran
    % TODO: Tabel Lampiran
    \newpage
    \appendix
    \addcontentsline{toc}{chapter}{Lampiran}
    \chapter*{Lampiran}
    \renewcommand\thesection{\Alph{section}}
    \section{Repository GitHub}
Hasil dari program bot permainan dapat diakses melalui repository GitHub yang berisi kode dan dokumentasi. Anda dapat mengunduh dan memeriksa versi terbaru dari program pada tautan berikut:
\href{https://github.com/haziqam/tubes1-IF2211-game-engine/releases/tag/v1.1.0}{GitHub Repository v1.1.0}.

\section{Video Penjelasan}
Penjelasan tambahan mengenai implementasi dan hasil dari program bot permainan dapat ditemukan dalam video yang diunggah ke Google Drive. Tautan untuk menonton video tersebut adalah:
\href{https://drive.google.com}{Video Penjelasan di Google Drive}.

\end{document}
